\documentclass[11pt]{article}
\usepackage{geometry}                % See geometry.pdf to learn the layout options. There are lots.
\geometry{letterpaper}                   % ... or a4paper or a5paper or ... 
%\geometry{landscape}                % Activate for for rotated page geometry
%\usepackage[parfill]{parskip}    % Activate to begin paragraphs with an empty line rather than an indent
\usepackage{graphicx}
\usepackage{amssymb}
\usepackage{epstopdf}
\usepackage{hyperref}
\DeclareGraphicsRule{.tif}{png}{.png}{`convert #1 `dirname #1`/`basename #1 .tif`.png}

\title{\textbf{Bike Sharing Stations in Bogota}}
\author{Lucia Perez Ramirez}
\date{}                                           % Activate to display a given date or no date

\begin{document}
\maketitle
\section{Introduction}
Bogota is ranked as the fifth most traffic congested city of the world. It takes the first place among South American capitals. According to the Global Traffic Scorecard of INRIX (2018), a Bogota citizen loses 75 hours a year in traffic jams. In fact, most Latin American cities have to deal with a rapid and unplanned urban growth which represents a major challenge in mobility and traffic dynamics.\\

Given this panorama, the promotion of the bicycle as a daily and safe mode of transport has become a common objective within the policies of sustainability and equity in large cities. The use of the bicycle not only reduces carbon emissions within big cities, it also helps to alleviate traffic congestion, decreases travel times, and favors people's health and wellbeing.\\

Nevertheless, Bogota is famous across the world for being a bike friendly city. It has a population of around eight million people and cycle paths covering more than 360km (220 miles) of the city’s surface. Almost 84,000 people use Bogota’s cycle route network every day, which only stands for around 1\% of the total population. This has made local government to ask themselves 'How to make the bicycle a daily and safe means of transport for most people?'.\\

I think that a bike-borrowing system would be appropriate for a city like Bogota in order to answer this question. This solution also deals with other concerns among citizens which include vandalism, parking or storage, and maintenance.\\

But then again another question arises and this is \textbf{which would be the ideal locations to put bike-sharing points within the city?}

\end{document} 
%\subsection{}
%\section{Data}

\end{document}  